\documentclass[a4paper,12pt]{article}

\usepackage[german]{babel}
\usepackage{blindtext}
\usepackage{charter}
\usepackage{index}

\begin{document}


\title{\Large{\textbf{Eine Analyse von Programmiersprachen gemäss spätem Wittgenstein}}}
\author{Lu Maltsis \\
Weinbergstrasse 105 \\
8006, Zürich \\
lu.maltsis@gmail.com \\
\textit{Eingereicht beim MNG, Rämibühl}
}
\date{Eingereicht am xx.xx.xxxx \\
\vspace{10mm}
Betreuung durch Christian Villiger}
\maketitle
\newpage


\section{Abstrakt}
\blindtext[1]


\tableofcontents{}


\section{Später Wittgenstein}
\blindtext[1]

\subsection{Die Semantik natürlicher Sprachen}
\blindtext[1]

\subsection{Privatsprachen}
\blindtext[1]

\subsection{Robinson Cruise}
\blindtext[1]


\section{Compiler}
\blindtext[1]

\subsection{Die Turing Maschine}
\blindtext[1]

\subsection{Die Programmiersprache C}
\blindtext[1]


\section{Mensch und Maschine}
\blindtext[1]

\subsection{Die chinesische Box}
\blindtext[1]

\subsection{Behaviorismus}
\blindtext[1]


\section{Vergleiche}
\blindtext[1]

\subsection{Das Entstehen von Programmiersprachen}
\blindtext[1]

\subsection{Das Paradox der Bedeutung}
\blindtext[1]

\section{Konklusion}
\blindtext[1]

\section*{Literaturverzeichnis}
\textbf{Kripke, Saul A. (1982)}: \textit{Wittgenstein on rules and private language: An Elementary Exposition.} Cambridge: Havard University Press. Deutsch: Übersetzt von Helmut Pape (1987): \textit{Wittgenstein über Regeln und Privatsprache. Eine elementare Darstellung.} Frankfurt am Main: Suhrkamp \\
\textbf{Newen, Albert und Markus A. Schrenk (2008)}: \textit{Einführung in die Sprachphilosophie.} Darmstadt: WBG \\
\textbf{Selzer, Edgard (2011)}: \textit{Denn der Mensch ist mehr als nur ein Computer. Warum die Turing-Maschine das WITTGENSTEIN'sche Sprachspiel nicht bewältigen kann.} Wien: Trauner \\
\textbf{Wittgenstein, Ludwig (1921)}: \textit{Tractatus Logico-Philosophicus.} in: \textit{L. Wittgenstein (1984): Werkausgabe in 8 Bänden. Band 1.} Frankfurt am Main: Suhrkamp \\
\textbf{Wittgenstein Ludwig (1953)}: \textit{Philosophische Untersuchungen.} in: \textit{L. Wittgenstein (1984): Werkausgabe in 8 Bänden. Band 1.} Frankfurt am Main: Suhrkamp \\




\end{document}