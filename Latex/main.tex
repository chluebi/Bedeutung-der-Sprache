\documentclass[a4paper,12pt]{article}

\usepackage[german]{babel}
\usepackage{blindtext}
\usepackage{charter}
\usepackage{index}
\usepackage{csquotes}
\usepackage{dirtytalk}

\begin{document}


\title{\Large{\textbf{Eine Analyse von Programmiersprachen gemäss spätem Wittgenstein}}}
\author{Lu Maltsis \\
Weinbergstrasse 105 \\
8006, Zürich \\
lu.maltsis@gmail.com \\
\textit{Eingereicht beim MNG, Rämibühl}
}
\date{Eingereicht am xx.xx.xxxx \\
\vspace{10mm}
Betreuung durch Christian Villiger}
\maketitle
\newpage


\section{Abstrakt}
\blindtext[1]


\tableofcontents{}


\section{Später Wittgenstein}
Während des Ersten Weltkriegs schrieb Ludwig Wittgenstein sein erstes Hauptwerk: Die \enquote{Logisch-philosophische Abhandlung}, Englisch: \enquote{Tractatus logico-philosophicus} (TLP)\footnote{Wittgenstein, Ludwig (1921), Seite 2} Hier werden allgemeine Sachverhalte (TLP, §2) als Konstrukte von Gegenständen (TLP, §2.1) bezeichnet. Die Sprache habe dann die Aufgabe diesen Gegenständen Namen zu geben (TLP, §3.22). Diese Philosophie wird als \enquote{früher Wittgenstein} bezeichnet. Der \enquote{späte Wittgenstein} widerspricht seinem Vorgänger explizit:
\begin{displayquote}
\enquote{[...] musste ich schwere Irrtümer in dem erkennen, was ich in jenem ersten Buche niedergelegt habe.}\footnote{Wittgenstein, Ludwig (1953), Seite 4}
\end{displayquote}
In den \enquote{Philosophischen Untersuchungen} (PU), dem Hauptwerk des späten Wittgensteins wird unter anderem die Bedeutung der Sprache mithilfe von Aphorismen erforscht. 
\begin{displayquote}
\enquote{[...] Jedes Wort hat eine Bedeutung. Diese Bedeutung ist dem Wort zugeordnet. Sie ist der Gegenstand, für welchen das Wort steht.} \hspace{1mm} (PU, §1)
\end{displayquote}


\subsection{Regeln}
Wie kann eine grammatische Regel (oder Regeln im allgemeinen) das Handeln festlegen? So ist doch jedes Regelfolgen mit einer Deutung der Regel verbunden. (PU, §198) \\
Findet man zum Beispiel einen Wegweiser vor, so wäre eine mögliche Deutung \enquote{Wenn Sie das auf dem Wegweiser angegeben Ziel erreichen wollen, dann können sie es am besten in Pfeilrichtung erreichen.} Doch eine entgegengesetzte Deutung wäre genauso möglich: \enquote{Wenn Sie das auf dem Wegweiser angegebene Ziel erreichen wollen, dann können Sie es am besten entgegen der Pfeilrichtung erreichen}. Würde man Regeln festlegen wie eine solcher Wegweiser zu deuten sei, müsste man diese Regeln auch wieder erklären etc. Es enstünde ein infiniter Regress. \footnote{Newen, Albert (2008), Seite 35} Demnach muss etwas anderes geben, was bestimmt wie wir eine Regel zu deuten haben.

\begin{displayquote}
\enquote{Einer Regel folgen, eine Mitteilung machen, einen Befehl geben, eine Schach-
partie spielen Sind Gepßogenbeiten (Gebräuche, Institutionen).} (PU, §199)
\end{displayquote}


\subsection{Die Bedeutung und Sprachspiele}
\enquote{Bedeutung ist Gebrauch} ist der Slogan dieser neuen Sprachphilosophie.(PU, §43) Einzelne Worte korrespondieren also nicht mit einem objektiven Sachverhalt in der Welt, sondern sind rein dadurch definiert, wie sie verwendet werden. Der Gebrauch der Sprache wird hierbei als \enquote{Sprachspiel} bezeichnet. (PU, §7)

\begin{displayquote}
\enquote{Das Wort 'Sprachspiel' soll hier hervorheben, daß das Sprechen
der Sprache ein Teil ist einer Tätigkeit, oder einer Lebensform. [...]
Es ist interessant, die Mannigfaltigkeit der Werkzeuge der Sprache
und ihrer Verwendungsweisen, die Mannigfaltigkeit der Wort- und
Satzarten, mit dem zu vergleichen, was Logiker über den Bau der
Sprache gesagt haben. (Und auch der Verfasser der Logisch-Philosophischen Abhandlung. ) \textit{(sic!)}} (PU, §23)
\end{displayquote}


\subsection{Privatsprachen}
\blindtext[1]

\subsection{Robinson Cruise}
\blindtext[1]


\section{Compiler}
\blindtext[1]

\subsection{Die Turing Maschine}
\blindtext[1]

\subsection{Die Programmiersprache C}
\blindtext[1]


\section{Mensch und Maschine}
\blindtext[1]

\subsection{Die chinesische Box}
\blindtext[1]

\subsection{Behaviorismus}
\blindtext[1]


\section{Vergleiche}
\blindtext[1]

\subsection{Das Entstehen von Programmiersprachen}
\blindtext[1]

\subsection{Das Paradox der Bedeutung}
\blindtext[1]

\section{Konklusion}
\blindtext[1]

\section*{Literaturverzeichnis}
\textbf{Kripke, Saul A. (1982)}: \textit{Wittgenstein on rules and private language: An Elementary Exposition.} Cambridge: Havard University Press. Deutsch: Übersetzt von Helmut Pape (1987): \textit{Wittgenstein über Regeln und Privatsprache. Eine elementare Darstellung.} Frankfurt am Main: Suhrkamp \\
\textbf{Newen, Albert und Markus A. Schrenk (2008)}: \textit{Einführung in die Sprachphilosophie.} Darmstadt: WBG \\
\textbf{Selzer, Edgard (2011)}: \textit{Denn der Mensch ist mehr als nur ein Computer. Warum die Turing-Maschine das WITTGENSTEIN'sche Sprachspiel nicht bewältigen kann.} Wien: Trauner \\
\textbf{Wittgenstein, Ludwig (1921)}: \textit{Tractatus Logico-Philosophicus.} in: \textit{L. Wittgenstein (1984): Werkausgabe in 8 Bänden. Band 1.} Frankfurt am Main: Suhrkamp \\
\textbf{Wittgenstein Ludwig (1953)}: \textit{Philosophische Untersuchungen.} in: \textit{L. Wittgenstein (1984): Werkausgabe in 8 Bänden. Band 1.} Frankfurt am Main: Suhrkamp \\
\textbf{Wittgenstein Ludwig (1956)}: \textit{Bemerkungen über die Grundlagen der Mathematik.} in: \textit{L. Wittgenstein (1984): Werkausgabe in 8 Bänden. Band 6.} Frankfurt am Main: Suhrkamp \\




\end{document}